切向加速度||
\vec{a}_\tau
== \frac{\mathrm{d}v}{\mathrm{d}t} \vec{\tau}_0
= \frac{\mathrm{d}^2 s}{\mathrm{d} t^2} \vec{\tau}_0

%%

法向加速度||
\vec{a}_n
== \frac{v^2}{\rho} \vec{n}_0
= \frac{v^2}{R} \vec{n}_0

%%

质点系动量定理||
\int_{t_1}^{t_2} \left( \sum_{i=1}^n\vec{F}_{i外} \right)\, \mathrm{d}t
== \sum_{i=1}^n m_i\vec{v}_{i2} - \sum_{i=1}^n m_i \vec{v}_{i1}

%%

保守力||
\oint_l \vec{F}_保 \cdot \mathrm{d} \vec{r}
== 0

%%

力矩||
\vec{M}
== \vec{r} \times \vec{F}

%%

角动量||
\vec{L}
== \vec{r} \times m\vec{v}

%%

角动量定理||
\vec{M}
== \frac{\mathrm{d}\vec{L}} {\mathrm{d} t}

%%

静电场中高斯定理||
\oint_S\vec{E}\cdot \mathrm{d} \vec{S}
== \frac{\sum q_i}{\epsilon_0}

%%

电场力||
\vec{F}
== \frac{1}{4\pi\epsilon_0}\frac{q_0q}{r^2}\vec{r}_0

%%

电通量||
\Phi_e
== \vec{E}\cdot\vec{S} = ES\cos \theta
= \oint_S\vec{E}\cdot\mathrm{d}\vec{S}

%%

a点电势能||
W_{pa}
== \int_a^\infty q_0\vec{E}\cdot \mathrm{d} \vec{l}

%%

a点电势||
U_a
== \frac{W_{pa}}{q_0}
= \int_a^\infty \vec{E}\cdot\mathrm{d}\vec{l}
= \frac{kQ}{r}
= \frac{Q}{4\pi\epsilon_0r}

%%

a,b电势差||
U_{ab}
==U_a - U_b
=\int_a^b \vec{E} \cdot \mathrm{d}\vec{l}

%%

导体达到静电平衡,其中任意两点U_{PQ}||
\int_P^Q\vec{E}\cdot\mathrm{d}\vec{l}
== 0

%%

安培环路定理||
\oint_l\vec{B}\cdot\mathrm{d}\vec{l}
== \mu_0\sum I_i

%%

运动电荷在磁场中受到的力||
f
== qvB
= m\frac{v^2}{R}

%%

毕奥-萨伐尔定律的微分形式,用于计算恒定电流产生的磁场||
\mathrm{d}\vec{B}
== k\frac{I\mathrm{d}\vec{l}\times\vec{r}}{r^3},k=\frac{\mu_0}{4\pi}

%%

载流直导线磁感应强度大小||
B
== \frac{\mu_0I}{2\pi a}

%%

载流圆周磁感应强度大小||
B==
\frac{\mu_0I}{2R}


